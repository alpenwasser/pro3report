There are a variety of tests that can be conducted to verify the performance and
correctness of our  device.  The  following is a list of test fixtures and their
expected results based on simulations.

\subsection{Test Fixtures}

\subsubsection{Trivial and Non-Trivial I-V Curves}

Reproduction of a custom I-V  curve is tested by using the test fixture depicted
in Figure \ref{fig:verification:iv-circuit-fix}.

\begin{figure}[th!]
    \centering
    \includegraphics[width=.4\textwidth]{images/sim/iv-curve-circuit.png}
    \caption{Test fixture for measuring the device's IV characteristic curve}
    \label{fig:verification:iv-circuit-fix}
\end{figure}

The  voltage  and  current  is  measured  using  different resistive loads. This
measurement is performed twice using two different I-V curve configurations; one
``simple''  curve (using a single-cell  model)  and  one  ``complicated''  curve
(using a  multi-cell  model  where  some of the cells are shaded). The resulting
measurements are then compared with  the  theoretical  curves  seen  in  Figures
\ref{fig:verification:iv-simple}  and  \ref{fig:verification:iv-complicated} and
thus the device's accuracy can be evaluated.

\begin{figure}[th!]
    \centering
    \begin{minipage}{.4\textwidth}
        \centering
        \includegraphics[width=.95\textwidth]{images/sim/iv-simple.png}
        \caption{Current/voltage characteristic curve (blue) and power/voltage characteristic curve (red) of a single-cell solar panel}
        \label{fig:verification:iv-simple}
    \end{minipage}
    \begin{minipage}{.4\textwidth}
        \centering
        \includegraphics[width=.95\textwidth]{images/sim/iv-complicated.png}
        \caption{Current/voltage characteristic curve (blue) and power/voltage charactersticic curve (red) of a multi-cell solar panel, some cells are shaded, causing the staircase}
        \label{fig:verification:iv-complicated}
    \end{minipage}
\end{figure}

\subsubsection{Ripple Voltage and Ripple Current vs Resistive Load}

Using LTspice\cite{ref:ltspice}, the circuit  of  our  regulator was modeled and
the  peak-to-peak  ripple  current  and  voltage  was  simulated using different
resistive  loads ranging from \SI{100}{\milli\ohm} to  \SI{1}{\kilo\ohm}  at  an
output  voltage of \SI{12}{\volt}. Figure \ref{fig:verification:ripple_sim} is a
plot  of  the   ripple   current   and   voltage   versus  the  resistive  load.

\begin{figure}[th!]
    \centering
    \includegraphics[width=.7\textwidth]{images/sim/ripple-vs-load.pdf}
    \caption{Simulation of the amplitude of the ripple voltage and ripple current vs different resistive loads, obtained using LTspice I-V's model of the LT3741}
    \label{fig:verification:ripple_sim}
\end{figure}

\begin{minipage}{0.5\textwidth}
    In  order   to  test  this   on  the   real  device,  a   constant  output
    voltage   of  \SI{12}{\volt}   is  programmed   and  a   potentiometer  is
    attached  to the  device's  output connectors,  as  illustrated in  Figure
    \ref{fig:verification:ripple_fix}

    The ripple  current is  measured over a  small sense  resistor $R_{sense}$
    using  an oscilloscope. The  ripple  voltage is  measured  using a  second
    channel of the oscilloscope by measuring the output voltage.
\end{minipage}
\begin{minipage}{0.5\textwidth}
    \centering
    \includegraphics[width=.9\textwidth]{images/sim/ripple-fixture.png}
    \captionof{figure}{Test fixture for measuring ripple voltage \& ripple current of the device}
    \label{fig:verification:ripple_fix}
\end{minipage}

The peak-to-peak ripple voltage  and current is measured for different resistive
loads ranging from \SI{100}{\milli\ohm} to \SI{1}{\kilo\ohm} and compared to the
simulated model.

\subsubsection{Power Absorbtion}

\begin{minipage}{0.5\textwidth}
    As  stated  in  the  specifications,  the device  must  have  the  ability
    to  sink   current  as  well  as   source  current.   In  order   to  test
    this,  the   device  is  programmed   to  output  \SI{12}{\volt}   and  is
    connected  in  series  with  a  current limiting  resistor  and  a  second
    power  source  set   to  a  higher  voltage,  as   illustrated  in  Figure
    \ref{fig:verification:power_absorbtion_fix}.

    The second  voltage source is  slowly increased until  \SI{3}{\ampere} are
    flowing  through  the  current  limiting  resistor  $R_1$  (which  can  be
    determined by  measuring the  voltage over said  resistor), all  while the
    device's  output  voltage  is  closely monitored. The  output  voltage  is
    expected to remain constant.
\end{minipage}
\begin{minipage}{0.5\textwidth}
    \centering
    \includegraphics[width=.9\textwidth]{images/sim/power-absorbtion-fixture.png}
    \captionof{figure}{Test fixture for measuring power absorbtion of the device.}
    \label{fig:verification:power_absorbtion_fix}
\end{minipage}



\subsubsection{Transient Response}

\begin{minipage}{0.5\textwidth}
    This  test is  used  to  determine the  reaction  speed  of the  regulator
    and  the  I-V  control  algorithm   when  switching  quickly  between  two
    extreme  resistive  loads.    The  device  is  programmed   to  mimic  the
    behaviour  of  a simple  solar  panel  model.   As illustrated  in  Figure
    \ref{fig:verification:transient_fix} the output  voltage is measured using
    an oscilloscope and  various resistive loads are switched on  and off over
    the output using a MOSFET and a signal generator.
\end{minipage}
\begin{minipage}{0.5\textwidth}
    \centering
    \includegraphics[width=.9\textwidth]{images/sim/transient-fixture.png}
    \captionof{figure}{Test fixture for measuring the transient response of the device}
    \label{fig:verification:transient_fix}
\end{minipage}

The frequency  of the signal  generator is  set to a  low value such  that the
device's output  voltage is  always stable  before the  load is  changed.  The
results of this  measurement will make a statement over  the propagation delay
of the  algorithm as well as  the time it  takes for the voltage  to stabilise
again.


\subsubsection{Reactive Loads}

In the coarse specification we  opted to test the device with various capacitive
and  inductive  loads  out  of  curiosity. During the course of the  project  we
learned that this is not an applicable test because reactive loads are primarily
used for testing the behaviour of power factor correction -- which, according to
the EN61000-3-2  regulations\cite{ref:pfc},  affect  devices consuming more than
\SI{75}{\watt},  meaning  that  the main power supply we use already implements
power factor correction and testing for this would  be meaningless. Furthermore,
the measurements we obtain  from  tests  using  reactive  loads would require an
immense  amount of additional research on our part before the results  could  be
interpreted in any  meaningful way. For this reason we have decided to omit this
test case.


\subsection{Measurement Results}

Unfortunately, we were unable to  operate the LT3741 for a period long enough to
conduct any of the measurements listed above.  The same regulator model (LT3741)
was  permanently  damaged  twice  in  a row. Please see the next section  for  a
detailed analysis on what we think the issue is and how we would proceed if more
time to do so were available.


\subsection{Analysis of the Issue}
\label{subsec:analysis_issue}

The  first regulator output the correct voltage  of  \SI{12}{\volt}  when  being
supplied  with  \SI{32}{\volt}.  Unplugging the device and re-plugging it into a
\SI{36}{\volt}  power  supply  instantly  damaged  it  permanently.  After  some
detailed measurements it  was  concluded  that  the  high-side driver inside the
LT3741  was somehow damaged and not operating as it should. Our  assumption  was
that -- since we were  operating  the device very closely to its maximum ratings
of  \SI{40}{\volt} -- during switching, the high-side MOSFET driver was  exposed
to transients exceeding the device's absolute maximum ratings  and thus damaging
the driver permanently.

The regulator was replaced  with a new one and the  supply voltage was lowered
from \SI{36}{\volt}  to \SI{28}{\volt} in  order to  give more leeway  for the
transient voltages. The new  regulator appeared to output  the correct voltage
of  \SI{12}{\volt}, so  a resistive  load  of \SI{80}{\ohm}  was connected  to
the  output of  the  regulator.  After  about  \SI{20}{\second} of  continuous
operation, the  output voltage  again dropped and  the device  was permanently
damaged. Unfortunately, we were  unable to capture some  vital measurements of
the transients  we were looking for  to confirm our suspicions  from the first
failure.  It is clear,  however, that the damage was not  caused by the device
overheating, as none  of the components were remotely warm  to the touch. This
seems to align with the transient theory.


\subsection{Simulating the Issue}

The physical  layout of the  regulator and the  MOSFETs is depicted  in Figure
\ref{fig:verification:long_traces_pcb}.   The  traces  connecting  the  Switch
(SW), High Gate (HG) and Low Gate  (LG) pins of the regulator to the switching
MOSFETs  are  fairly  long  (>\SI{2}{\centi\metre}). Most  of  the  time,  the
parasitic inductances and capacitances  are negligible. In this case, however,
it turned out that they are not.

The series  inductance  and  series  resistance  is  calculated using an on-line
calculator\cite{ref:trace_inductance}\cite{ref:trace_resistance}   (the   width,
length,    and    thickness    is    known   to    be    \SI{0.4}{\milli\metre},
\SI{2}{\centi\metre}  and \SI{35}{\micro\metre} respectively) and added  to  the
simulation   model  seen  in  Figure   \ref{fig:verification:long_traces_model}.

\begin{figure}[th!]
    \centering
    \begin{minipage}{.6\textwidth}
        \centering
        \includegraphics[width=.9\textwidth]{images/pcb/long-traces.png}
        \caption{Physical layout of MOSFETs and regulator}
        \label{fig:verification:long_traces_pcb}
    \end{minipage}
    \begin{minipage}{.38\textwidth}
        \centering
        \includegraphics[width=.9\textwidth]{images/sim/lt3741-transients-circuit-real.png}
        \caption{Simulation model with long traces}
        \label{fig:verification:long_traces_model}
    \end{minipage}
\end{figure}

The voltage on the SW pin of the regulator  is  simulated  and  plotted.  Figure
\ref{fig:verification:long_traces_simulation}  compares  the   more   accurately
modelled curve (in blue, labelled \emph{sw\_real})  with  the ideal model of the
voltage on the SW pin (in yellow, labelled \emph{sw\_ideal}). The  two red lines
represent  the  expected  maximum  voltage  (\SI{28}{\volt})  and  the  device's
absolute maximum rating on the SW pin (\SI{40}{\volt}).

\begin{figure}[th!]
    \centering
    \includegraphics[width=.7\textwidth]{images/sim/lt3741-transients-sim.png}
    \caption{Simulation of voltage on SW pin. \emph{Yellow:} Idealised, \emph{blue:} actual, \emph{green}: actual, with series resistors as a fix.}
    \label{fig:verification:long_traces_simulation}
\end{figure}

It is clear that these small parasitic  inductances have a huge impact. The real
curve  (blue) exhibits ringing  with  an  amplitude  overshooting  the  expected
maximum voltage by  almost  \SI{12}{\volt}! It is therefore very likely that the
high driver was destroyed due to an over-voltage event.

The immediate solution to this problem  is  to add resistors in series with each
of  the  critical  connections  to  dampen  the  ringing.  The   effect   of   a
\SI{3.3}{\ohm} resistor in series  with  each  connection  is  plotted in figure
\ref{fig:verification:long_traces_simulation}     (green     curve,     labelled
\emph{sw\_fix}). When compared  to  the  ideal  and real curves, we see that the
peak voltage has been reduced to a much less dangerous level.

The more  optimal solution  (more elegant  and cheaper) is  to modify  the PCB
layout to drastically reduce the length of these traces.
