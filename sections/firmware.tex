%\begin{itemize}
%    \item
%        Wie wird das mathematische Modell in Software implementiert?
%    \item
%        Wie wird die Kommunikation mit einem PC implementiert? (Protokoll)
%    \item
%        Event-System
%    \item
%        User-Interface am Ger\"at
%\end{itemize}


\subsubsection{Constraints for the uC programm}

The chosen  microcontroller dsPIC33ep  provides 16kB  of ROM  and 2kB  of RAM,
12bit resolution  in the ADC  and DAC and up  to 60 MIPS  computing power. The
anti-aliasing filter of  the ADC is set  to 5khz, which means that  we need to
sample with at least  \SI{10}{\kilo\hertz}. However the LT3741 regulator takes
about \SI{300}{\micro\second} to  adjust its output voltage --  3 times longer
than our  sample rate. To remedy this  we use Oversampling by  factor 4, which
has  the nice  benefits  of giving  us  an additional  bit  of resolution  and
making the anti-aliasing  filter simpler. The main routine  for the adjustment
of  the  regulator  should  not  use  more than  25\%  of  the  CPU  time  and
must  run every  \SI{400}{\micro\second},  so  it must  not  take longer  than
\SI{100}{\micro\second} or about 6000 Instructions to finish one calculation.


\subsubsection{I-V Curve and Operating Points}
\label{subsec:iv-curve-operating-points}

The  I-V  curve  from  \eqref{eq:I-V} is implemented using a fixed point  library
supplied by Microchip. Since this library also provides an exponential function,
the implementation is made straightforward.

To  find the  target  operating point,  we calculate  the  load resistance  by
dividing the voltage  on the output by  the current on the  output $R_{load} =
U_{is} /  I_{is}$. Then we draw the  load line $I =  f(U) = U /  R_{load}$ and
find  the intersection  with the  I-V  curve of  the model.   This process  is
pictured in Figure \ref{fig:model:approx}.  The intersection is found by using
a binary search algorithm, which is converging slower than newton's method but
more stable -- Newton's method can fail if $V_T$ in \eqref{eq:I-V} is small.

If we want to use multiple curves to model the staircase characteristic shown in
Figure \ref{fig:model:steps} we  need to determine which curve is applicable for
a given load. This process can be made less CPU intensive if we precalculate the
resistance-thresholds by finding the intersections between two curves.  There is
always exactly one intersection if  $V_{oc1} > V_{oc2}$ and $I_{sc1} < I_{sc2}$.

\begin{minipage}{0.5\textwidth}
	\center
    \includegraphics[width=.95\textwidth]{images/model/approx.png}
    \captionof{figure}{Actual and target operating points}
    \label{fig:model:approx}
\end{minipage}
\begin{minipage}{0.5\textwidth}
	\center
    \includegraphics[width=.95\textwidth]{images/model/threshold.png}
    \captionof{figure}{Threshold resistance for selecting a curve}
    \label{fig:model:threshold}
\end{minipage}
