When designing  a PCB --  especially when a  switch-mode power converter  is a
central  component to  the  design --  the location  of  components and  their
routing (electrical connections) can be  critical for correct operation.  Some
of the more important items that were considered are listed here.
\begin{itemize}
    \item
        High frequency, high power loops are routed as tightly as possible.
    \item
        Sensitive, high impedance traces are kept separate from other signals
        and routed as differential pairs where necessary.
    \item
        Digital logic is kept separate from analog and high power circuitry.
    \item
        Power rails and their bypass capacitors need to be placed intelligently.
    \item
        Larger copper areas can be used to meet heat dissipation requirements.
    \item
        The positioning of mechanical parts can be annoying because they take up
        way more space than what one might initially, naively, expect.
\end{itemize}

\begin{minipage}{0.5\textwidth}
    Figure \ref{fig:pcb:partitioning}  shows how  our \SI{60x60}{\milli\meter}
    printed circuit board is partitioned. The ground plane has been split from
    top to  bottom, physically  separating partition  \emph{A} from  the other
    three partitions.  The LT3741 is placed in the centre where top and bottom
    planes join  because it must  communicate both  with the digital  logic as
    well  as  the  high  power  circuitry. Partition  \emph{A}  contains  high
    power/high frequency  components, such as  the two switching  MOSFETs, the
    inductor, and the output capacitors,  whereas the other partitions contain
    more  sensitive circuitry  (in particular  partition \emph{C}). The  split
    helps minimising crosstalk.

    Partition \emph{C}  contains the micro controller,  LCD header, push/twist
    button header and other digital components.

    Partition \emph{B} contains the  components responsible for generating the
    three voltage  rails discussed in section  \ref{sec:voltage_rails}. It was
    placed at  the bottom of  the board where the  power input is  located, to
    minimize the trace  lengths required for power distribution  on the board,
    and  it  was also  placed  far  away  from  partition \emph{B}  such  that
    interference with the digital circuitry is minimized.

    Partition \emph{D} is  electrically isolated from the rest  of the circuit
    and contains  the components  responsible for  USB communication.   It was
    isolated because  the voltage potential of  a connected USB device  may be
    different than the potential our device is running on.

    Figure  \ref{fig:pcb:loops1}  shows the  two  critical  loops where  short
    intervals  of high  amounts  of  current flow  in  this design. The  first
    (green) loop is  active when the switching MOSFET $V_2$  is conducting and
    transferring  charge  from  the  input  bypass  capacitors  $C_3$,  $C_4$,
    $C_{12}$ and $C_1$ through the resistor $R4$ into the inductor $L_1$.
\end{minipage}
\begin{minipage}{0.5\textwidth}
    \centering
    \includegraphics[width=.9\linewidth]{images/pcb/partitioning.png}
    \captionof{figure}{Partitioning scheme of component groups.
        \textbf{A:} High power components
        \textbf{B:} Voltage rails
        \textbf{C:} Digital logic
        \textbf{D:} Isolated transceiver logic
    }
    \label{fig:pcb:partitioning}
    \centering
    \includegraphics[width=.9\textwidth,trim=0 25mm 0 0,clip]{images/circuit/schematic_high_current.png}
    \captionof{figure}{Critical high current, high frequency loops in the schematic. Blue indicates the current path of the first critical loop, green the second.}
    \label{fig:pcb:loops1}
\end{minipage}

The second (blue) loop is active when the switching mosfet $V_3$ is conducting
and transferring  charge from  the inductor $L_1$  through the  resistor $R_4$
into the output bypass capacitors $C_{15}$, $C_{11}$, $C_7$ and $C_8$.




%\begin{figure}[h!]
%    \center
%    \includegraphics[width=.3\textwidth]{images/pcb/buck2.png}
%    \caption{High current, high frequency loops are routed as tightly as possible}
%    \label{fig:pcb:loops2}
%\end{figure}
